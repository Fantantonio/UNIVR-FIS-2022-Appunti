\section{UML}
\label{sec:05_uml}
Così come i requisiti anche il design deve essere documentato.
Lo standard \acrfull{UML} è nato appunto con questo obiettivo.\\
Negli anni 80 è nata l'esigenza di avere un linguaggio per la modellazione ma il tentativo di creare un linguaggio condiviso da tutti è stata fallimentare.\\
L'UML può essere utilizzato in vari contesti, ad esempio per abbozzare soluzioni, ma anche per fornire il dettaglio implementativo di un sistema.\\
Ci sono varie tipologie di diagrammi UML, non tutti vengono utilizzati correntemente, ma è importante conoscerli.
I diagrammi si possono dividere in tre categorie:
\begin{itemize}[noitemsep]
    \item \textbf{Modelli strutturali}: descrivono l'organizzazione e le relazioni tra le varie parti del sistema. Ci sono delle strutture più statiche e dinamiche che descrivono lo stato a run-time. Il class diagram è sicuramente quello più utilizzato e descrive le varie entità del sistema, con le relative relazioni. Ogni classe è rappresentata da un rettangolo. All'interno ci sono il nome della classe, i vari attributi e metodi. Ogni attributo o metodo ha una visibilità (+ pubblica, - privata, \# package, $\sim$ protected). Le classi rappresentano una vista statica del sistema, mentre gli oggetti sono le istanze di una classe e sono dinamici. Le classi possono anche essere in relazione tra di loro (associazione, generalizzazione, aggregazione, composizione). Il class diagram è utilizzato quando si vuole rappresentare la natura statica del sistema e le relazioni tra le varie componenti. Si possono anche attribuire le responsabilità alle varie classi; se si nota che una classe ne ha troppe è da valutare l'idea di spezzarla in più classi. Allo stesso modo, se una classe non ha responsabilità, è da valutare l'idea di rimuoverla
    \item \textbf{Modelli di interazione}: descrivono le interazioni tra le varie parti del sistema
    \item \textbf{Modelli di comportamento}: descrivono come il sistema deve agire
\end{itemize}