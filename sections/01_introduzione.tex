\section{Introduzione}
\label{sec:01_introduzione}
Il software è un elemento indispensabile per le moderne economie, per gli stati e le organizzazioni.
Il problema del software è che è astratto ed intangibile quindi è difficile identificarne le proprietà di qualità e robustezza.\\
Di fatto quando si sviluppa il team è composto da persone ed ognuna di esse conosce molto bene una parte e poco tutte le altre parti.
Questo solitamente causa grossi problemi che riguardano principalmente la qualità del software che può degenerare velocemente trasformandolo in qualcosa di molto difficile da mantenere, gestire ed utilizzare.

L'ingegneria del software è una serie di tecnologie e teorie per supportare lo sviluppo professionale di software.\\
Se parliamo di quanto costa il software cioè quanto costa svilupparlo, sappiamo che il costo è di vari ordini di grandezza superiore rispetto al costo dell'hardware.
Tanto più ora che, nella maggior parte dei casi, il costo dell'hardware è diventato il prezzo dell'utilizzo di un servizio cloud.\\
Dalle statistiche si evince che il costo per la manutenzione del software è di gran lunga superiore al costo per il suo sviluppo.
La manutenzione significa anche aggiornamenti che si rendono necessari per l'evoluzione delle normative e delle leggi alle quali il software deve sottostare.\\
Ci possono anche essere problemi legati al cambiamento del contesto nel quale il software è eseguito per il quale si necessita anche del cambiamento del sistema software.\\
Un esempio di software in esecuzione da decenni sono i software legacy delle banche; il costo di manutenzione in questo caso è di gran lunga più alto rispetto a quello della sua produzione.\\
La disciplina dell'ingegneria del software ha come obiettivo quello di supportare uno sviluppo software che sia cost-effective, cioè che possa soddisfare gli obiettivi di costo che possono essere sostenuti dal cliente.\\
Secondo una ricerca, solo circa il 29\% dei progetti vengono terminati con successo (entro i tempi ed entro il budget).
La maggioranza dei progetti, invece, ha raggiunto dei compromessi di tempi e costi.
In alcuni casi si è ritardata la scadenza, in altri si sono accettati costi maggiori, in altri ancora si è accettato di avere un sottoinsieme di funzionalità rispetto a quelle richieste inizialmente.\\
I problemi tipici dei progetti software sono:
\begin{itemize}[noitemsep]
    \item \textbf{Stima dei costi}: i costi effettivamente sostenuti risultano superiori a quelli stimati all'inizio. Questo perché stimare i costi del software è impossibile data la natura intangibile dello stesso. A meno che non si abbia l'esperienza di realizzazione di un sistema molto simile da utilizzare come confronto.
    \item \textbf{Stima del tempo}: i costi possono anche essere stimati correttamente, ma il software viene consegnato oltre i tempi definiti dal contratto. In questo caso il committente potrebbe anche venire meno al pagamento del contratto.
    \item \textbf{Errata interpretazione}: un altro problema è che il software consegnato potrebbe essere disallineato rispetto alle richieste fatte dal committente.
    \item \textbf{Qualità del codice}: il software potrebbe anche avere una scarsa qualità, in termini di scrittura del codice (che pone il problema della manutenzione successiva, oppure in termini di utilizzo (occupa troppa memoria, si blocca spesso, ecc.).
    \item \textbf{Poor design}: l'ultimo problema è che gli utenti finali potrebbero non essere pienamente soddisfatti dal software che è stato consegnato per via dell'usabilità, della velocità o della user-experience.
\end{itemize}
Le motivazioni per cui queste problematiche insorgono sono legate alla difficoltà intrinseca della complessità del problema, come appunto la stima dei costi, ma anche dei problemi tecnologici come il linguaggio di programmazione da utilizzare, l'infrastruttura su cui girerà il software (cloud o altro).\\
Ci sono tuttavia ragioni tangibili che rendono difficile lo sviluppo del software.
Una delle prime motivazioni è la comunicazione.
Tipicamente ci sono dei problemi di comunicazioni tra il tecnico e il committente che detta i requisiti del software.
Questo per via della discrepanza di conoscenze tra lo stakeholder e l'ingegnere del software incaricato della creazione del prodotto.
Questo è il problema più importante che provoca l'inutilità del software per via dei requisiti raccolti in maniera incorretta e sommaria.\\
Un altro problema è il capire quali sono le vere esigenze dell'utente finale, che potrebbe non volere un'interfaccia raffinata, ma piuttosto un sistema che risolva i suoi problemi di produttività quotidiana.\\
Ci possono essere poi difficoltà nel comprendere qual è il dominio dell'organizzazione all'interno della quale l'ingegnere va ad operare.\\
Infine altri problemi possono essere legati all'organizzazione e alla gestione delle persone che lavorano su di esso come la difficoltà da parte di alcune risorse umane di lavorare in team.

Il termine \textbf{ingegneria del software} è stato coniato alla Nato Conference, l'attuale \acrfull{ICSE}\footnote{http://www.icse-conferences.org}, uno dei più grandi incontri sull'ingegneria del software.
In queste conferenze vengono presentati nuovi strumenti e metodologie per la gestione del software.
Negli ultimi anni, in questi incontri, è tema comune quello dell'intelligenza artificiale legata alla gestione e manutenzione del software.\\
Il termine ingegneria è legato alla volontà iniziale di strutturare la creazione del software come un processo produttivo ingegneristico ad esempio la creazione di un'automobile.
L'idea era quella di tradurre ciò che era già noto in meccanica, edilizia ecc nel mondo dello sviluppo software.
In questo modo vengono elencati tutti gli aspetti che sono legati allo sviluppo del software, dalla raccolta dei requisiti, all'architettura ad alto livello, alla produzione vera e propria del software.\\
Se nelle progettazione di automobili si guardano tutti i dettagli, anche in questo caso si devono guardare i dettagli di sviluppo per evitare possibili futuri errori.\\
Quindi l'obiettivo dell'ingegneria del software è quello di ottenere un software che abbia la qualità richiesta, che rispetti le scadenze ed i vincoli di costo prefissati.\\
Ovviamente questo impone di dover raggiungere dei compromessi ad esempio per essere in linea con tempi e costi si scenderà a compromessi su qualità e performance.\\
I professionisti attivi nell'ambito dell'ingegneria del software devono essere in grado di produrre sistemi affidabili e funzionanti in maniera rapida.
Queste due cose sono contrastanti, infatti più brevi sono le tempistiche di produzione e più bassi sono i costi, più il codice sviluppato sarà difettoso e di bassa qualità.
L'importante sempre e comunque è soddisfare le richieste dal committente.\\
Ci sono varie metodologie di sviluppo per riuscire, con i giusti compromessi, a produrre un software stabile scegliendo di volta in volta quella giusta in base a ciò di cui si necessita.\\
Una volta raccolti i requisiti si deve progettare il software e discutere il progetto con il team per assicurarsi che la conoscenza sia condivisa.\\
Come professionisti dell'ingegneria del software si viene a contatto anche con gli aspetti etici di questa professione.
Ad esempio, ci si aspetta che l'ingegneria del software abbia un comportamento etico e responsabile.
Prima di tutto la confidenzialità, anche senza che venga richiesta esplicitamente con la stipulazione di un accordo di riservatezza.
Come ingegneri del software non si devono ventilare competenze che non si possiedono, si è tenuti a rispettare le leggi della proprietà intellettuale del software e si è tenuti a non abusare delle risorse informatiche alle quali si ha accesso.
Ci sono associazioni internazionali che raccolgono vari ingegneri del software nel mondo.