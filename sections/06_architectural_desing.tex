\section{Architectural desing}
\label{sec:06_architectural_desing}
Dopo aver raccolto i requisiti, le funzionalità ed i servizi che il progetto dovrà fornire si passa alla progettazione del software.
L'obiettivo della progettazione dell'architettura è quello definire come il software debba essere organizzato.
È una fase intermedia che tra la definizione dei requisiti e la progettazione nel dettaglio del sistema.
È importante fare queste valutazioni ad alto livello per vari motivi:
\begin{itemize}[noitemsep]
    \item Migliore comunicazione con lo stakeholder. Parlare ad alta voce della struttura ad alto livello può aiutare a non scordare dei dettagli
    \item Evidenziare i problemi a livello architetturale così da tenerne conto già in principio
    \item Concordare il riutilizzo di software già disponibile o utilizzato in precedenza
\end{itemize}
L'utilizzo della progettazione architetturale si concentra nel facilitare la discussione sul design del sistema e la sua documentazione ad alto livello.\\
La sua produzione non è tanto un passo tecnico, quanto un processo che richiede di prendere delle decisioni.
Per esempio, come i vari artefatti software verranno distribuiti nell'hardware, l'utilizzo di pattern architetturali che vengono riutilizzati nel tempo per realizzare determinati tipi di sistemi, la discussione su queste architettura permette inoltre di capire se è possibile soddisfare i requisiti non funzionali.\\
Il sistema viene scomposto nei suoi componenti principali e così da capire se ognuno di essi ha una responsabilità definita, ambigua o distribuita su più componenti (questo causa problemi).\\
Le conseguenze che una scelta di un'architettura può avere sono:
\begin{itemize}[noitemsep]
    \item Performance
    \item Sicurezza
    \item Safety del sistema finale
    \item Affidabilità
    \item Manutenibilità
\end{itemize}
Scelta l'architettura è possibile fornire diverse rappresentazioni/viste della stessa.
\begin{itemize}[noitemsep]
    \item \textbf{Vista logica}
    \item \textbf{Vista di processo}: mostra quali sono e le interazioni tra i vari processi
    \item \textbf{Vista di sviluppo}: scompone il sistema in parti che verranno assegnate a team o sviluppatori distinti
    \item \textbf{Vista fisica}: permette di capire come i componenti potranno essere installati sui diversi dispositivi o i diversi host
\end{itemize}
È importanti rifarsi a dei pattern architetturali per avere una terminologia condivisa e universalmente accettata perché sono soluzioni buone a problemi ricorrenti.
Avere dei pattern che rappresentano le best practice nelle scelte architetturali aiuta molto nella fase di progettazione.
I pattern possono anche non fornire una soluzione adatta al problema da risolvere, ma a quel punto è compito dell'ingegnere integrarlo con qualche funzionalità in più per adattarlo meglio alle esigenze del caso.
Nella realtà infatti i problemi non sono così semplici e ci si trova spesso a dover più che altro prendere spunto dai pattern adattando l'architettura scelta al caso in oggetto.

\subsection{Model-View-Controller}
È uno dei pattern più utilizzati per la programmazione web ed è abbreviato con l'acronimo MVC.
Il pattern architetturale in questione prevede tre ruoli:
\begin{itemize}[noitemsep]
    \item \textbf{Model}: responsabile della gestione e della persistenza dei dati. Si interfaccia col \acrfull{DB}
    \item \textbf{View}: è responsabile di costruire l'interfaccia grafica e presentarla all'utente finale
    \item \textbf{Controller}: è responsabile di implementare la logica di business. Deve essere in grado di catturare le azioni dell'utente e rispondere alle richieste che arrivano
\end{itemize}
Per comprendere meglio si consideri ad esempio una webapp.
Il web browser riceve una pagina html costruita dal componente view e tutte le interazioni che l'utente con il web browser rappresentano un evento grafico che il web browser invia al controller il quale riceve gli input e li elabora per poi eseguire l'update sul \acrshort{DB} attraverso il componente model.\\
L'MVC è utilizzato in Java anche per applicazioni desktop con grafica e per applicazioni Android.\\
L'MVC andrebbe usato quando i modi di interagire o visualizzare i dati sono molteplici (varie schermate) poiché è meglio disaccoppiare i dati dalla loro presentazione così che si possano creare più modalità di visualizzazione dei dati sfruttando la stessa interfaccia con il \acrshort{DB}.
Tuttavia il disaccoppiamento introduce un po' di complessità nel codice.

\subsection{Layered architecture}
Questo pattern architetturale prevede la presenza di più strati uno sopra l'altro.\\
Si parte dal livello \virgolette{core}, centrale, si costruisce sopra di esso cosicché lo strato sottostante fornisca dei servizi allo stato sovrastante.\\
Le comunicazioni sono solo tra strati adiacenti in modo da permette di utilizzare una strategia di sviluppo incrementale.
Per cambiare uno strato bisogna però implementare tutte le interfacce del precedente così come sono implementate dallo strato da sostituire.
Si prenda come esempio un sistema informativo; il primo stato comprende il sistema operativo ed il database, sopra c'è lo strato che implementa il core business, poi uno strato che si occupa dell'autenticazione e del management della user interface ed infine lo strato più in alto che gestisce l'interfaccia grafica.\\
Tale archawitettura a strati è utilizzata dal sistema ISO/OSI per le comunicazioni di rete.\\
Anche il sistema operativo Android utilizza un'architettura a strati.\\
Questo tipo di architettura è molto utilizzata soprattutto nella costruzione di servizi sopra altri servizi pre-esistenti ma si potrebbe anche utilizzare quando lo sviluppo dei vari strati è in capo a team di sviluppo diversi.
È invece opportuno utilizzarla in caso di requisiti di sicurezza a più livelli poiché possono essere eseguiti controlli sulla sicurezza a più livelli, ed è possibile rimpiazzare uno o più layer fintanto che soddisfano la stessa interfaccia.\\
Ci sono tuttavia degli svantaggi; non è sempre possibile raggiungere un livello di isolamento totale, perciò bisogna permettere dei \virgolette{buchi} per fare dei salti di layer.
Un altro problema tipico riguarda le performance perché, ad esempio, per fare una chiamata dal layer più in alto a quello più in basso viene eseguita una lunga catena di chiamate.

\subsection{Repository architecture}
Il pattern architetturale di tipo repository prevede un componente centralizzato che mantiene una versione autorevole dei dati e vari componenti satelliti che interagiscono solo col componente centrale.\\
I dati possono essere generati anche dai componenti satelliti e vengono salvati nel repository centrale che si occupa di aggiornare anche gli altri componenti satelliti.
Un esempio di repository è quello usato dai sistemi di version control (Git).
I repository sono utilizzati quando ci sono grandi volumi di dati che devono rimanere consistenti a lungo nel tempo cioè nei sistemi \textbf{data-driven}.\\
Il vantaggio è che può esserci totale indipendenza tra i vari componenti satellite. 
I dati sono tipicamente consistenti perché c'è un solo punto in cui sono mantenuti ma di contro c'è il fatto che esiste un singolo point-of-failure ed anche uno svantaggio legato all'inefficienza derivante dal fatto che tutte le comunicazioni devono passare per il componente centrale il quale potrebbe trasformarsi in un collo di bottiglia.

\subsection{Client-Server}
Il pattern client-server prevede che ci siano uno o più componenti ad avere il ruolo del server e che forniscono servizi ai vari client.
I server sono connessi a una rete attraverso la quale i client richiedono i servizi.
Le richieste e le risposte sono gestite da un protocollo di comunicazione che, in quasi tutti i casi, è l'\acrfull{HTTP}.\\
Questa struttura è consigliata quando si hanno dati messi a disposizione da una serie di servizi dislocati sul territorio.\\
Avere più server permette di replicarli scalando in modo da prevedere un re-indirizzamento e gestire meglio il carico.
Tra gli altri vantaggi c'è il fattore della scelta della posizione dei server che per ridurre la latenza della comunicazione possono essere scelti il più vicino possibile al client, in caso in cui uno dei server avesse problemi e non fosse più disponibile il servizio non verrebbe negato ai client.\\
Tra gli svantaggi si considerino la difficoltà di prevedere le performance perché dipendono dal tipo e dal carico della rete ed il fatto che alcuni server potrebbero essere gestiti da terze parti.

\subsection{Peer-to-peer}
L'architettura peer-to-peer generalizza l'architettura client-server \virgolette{mescolandola}.
Ogni componente dell'architettura gioca il ruolo sia di client che di server nel senso che entrambi i ruoli sono giocati dallo stesso componente e tutti i componenti sono sullo stesso livello.\\
Un vantaggio di queste reti è che scalano molto bene.
Se uno dei peer non dovesse funzionare, il servizio verrebbe richiesto a un altro componente senza intermediazioni o necessità di operazioni manuali.
Un esempio di questa architettura è il protocollo \textbf{bitTorrent}.\\
Oltre ai client-server è necessario anche un \textbf{tracker} che mantiene traccia di quali \textbf{peer} afferiscono all'architettura.\\
Un altro vantaggio è la velocità perché dipende dalla struttura della rete cioè un file è suddiviso in varie parti e si può richiedere una parte da ciascun peer.
L'architettura peer-to-peer è utilizzata anche da bitcoin.

\subsection{Pipe and filter}
Pipe and filter è un'altra architettura che prevede la partecipazione di vari componenti ad una pipeline di generazione e gestione dei dati.\\
Un componente genera i dati, li passa ad un altro che li filtra e poi ad un terzo che li analizza.
Infine l'ultimo componente visualizza i dati elaborati.\\
Tipicamente questa architettura è utilizzata quando si devono gestire ed analizzare dei dati e non è appropriata per attività interattive in cui è previsto l'intervento dell'utente.\\
È inoltre opportuno utilizzarla quando l'input ha bisogno di vari step di processing prima di essere trasformato in output.
Questa architettura rende molto facile capire le responsabilità dei componenti ed è facile anche modificarle perché è molto intuitivo aggiungere nuovi filtri sia in maniera sequenziale che in maniera concorrente.\\
Ci sono però anche degli svantaggi; il formato dei dati deve essere deciso prima di iniziare l'implementazione del sistema, ogni componente deve leggere i dati e farne il parsing per estrarre le informazioni prima di poterle elaborare e successivamente deve ricomporli in un formato di scambio (es. \acrfull{XML} o \acrfull{JSON}).
