\section{Risk}
\label{sec:09_risk}
È un argomento trasversale, ma molto importante per l'ingegneria del software.
I progetti software sono difficili da condurre e da portare a termine con successo perché devono soddisfare i vincoli di organizzazione, di budget e di tempo.\\
Per raggiungere questi obiettivi ci sono delle metodologie che permettono di agevolare il lavoro.\\
Sicuramente se non si adotta una buona strategia di management, il progetto è destinato a fallire, tuttavia non si può essere al 100\% sicuri che adottando un buon management il risultato sarà positivo, quanto meno però si è sulla buona strada.
A differenza di altri tipi di progetti, il software è intangibile, quindi è difficile capire quale sia esattamente lo stato di avanzamento oltre al fatto che ogni progetto software è di per sé unico e richiede la risoluzione di nuovi problemi e che il processo adottato può variare in base all'organizzazione nella quale si opera.\\
Ci sono numerosi fattori che possono influenzare il progetto e il tipo di project management che si decide di implementare:
\begin{itemize}[noitemsep]
    \item \textbf{La dimensione dell'azienda}: in aziende piccole le persone sono più controllate ed è possibile anche una comunicazione informale tra le persone
    \item \textbf{Il tipo di stakeholder}: il software viene scritto in base allo stakeholder di riferimento. Se il software da realizzare andrà utilizzato all'interno della stessa azienda sarà molto più facile comunicare con gli stakeholder
    \item \textbf{La dimensione del software}
    \item \textbf{Il tipo del software}: software critico ha tendenzialmente bisogno di un processo di sviluppo abbastanza specifico in cui ogni scelta deve essere motivata
    \item \textbf{La cultura dell'organizzazione}: l'azienda è disposta a prendersi dei rischi oppure no
    \item \textbf{Il processo di sviluppo}: avrà un impatto sul tipo di gestione del progetto
\end{itemize}
Per gestire un progetto si seguono alcune fasi:
\begin{itemize}[noitemsep]
    \item \textbf{Project planning}: definizione di una pianificazione per il progetto, date delle release ecc, quando verrà consegnato il progetto
    \item \textbf{Risk management}: definizione delle procedure da adottare in caso si manifestino dei problemi
    \item \textbf{People management}
    \item \textbf{Reporting}: consegna dei report sullo stato di avanzamento del progetto
    \item \textbf{Proposal writing}: definizione di una proposta di progetta, una volta finita la progettazione, da consegnare al decision maker
\end{itemize}
Per quanto riguarda il Risk management si possono individuare varie fasi.

\subsection{Risk identification}
È la parte più corposa dal punto di vista temporale, perché bisogna identificare quali sono i rischi che il progetto può incontrare.\\
I rischi possono essere di stima sull'effort, sul tempo, sulle persone necessarie e possono compromettere l'intero progetto perché comportano errori di stima del costo.
Vanno considerati anche i rischi organizzativi (potrebbero derivare dall'azienda, come problemi finanziari, problemi di budget), rischi legati al personale (potrebbe essere difficile acquisire le persone con la competenze necessarie a lavorare al progetto oppure alcuni sviluppatori potrebbero andare in malattia), rischi legati ai requisiti (mancanze nella raccolta dei requisiti che quindi vanno rivisti e conseguentemente va riscritto anche il codice corrispondente), rischi legati alla tecnologia (ad esempio l'utilizzo di un \acrshort{DB} che non è in grado di soddisfare tutte le richieste) ed infine rischi legati agli strumenti che vengono utilizzati per facilitare il lavoro come generatori di codice o altro.

\subsection{Risk analysis}
Una volta identificati i rischi, essi vanno analizzati perché non sono tutti ugualmente importanti.\\
Sono quindi da assegnare le priorità in base all'importanza del rischio.
La criticità viene rappresentata in due dimensioni: la probabilità che il rischio si manifesti e la \textbf{seriousness} cioè l'impatto che avrebbe sul progetto.
Tipicamente si utilizza la scala (molto bassa, bassa, alta, molto alta) per rappresentare le criticità ed essendo una scala qualitativa è abbastanza soggettiva, pertanto spetta all'ingegnere essere il più oggettivo possibile.\\
Per ogni rischio quindi si deve associare un valore per la probabilità e la seriousness.

\subsection{Risk planning}
Dopo aver analizzato i rischi è bene cercare di capire quali sono le azioni da intraprendere nel momento in cui questi rischi si dovessero manifestare.\\
Per prima cosa bisognerebbe cercare di evitare il rischio, cambiando il progetto e cercando di evitare che si manifesti.\\
Se non è possibile evitarlo, si cerca di minimizzarne l'impatto.
Importante è la creazione di un piano di contingenza (es. rischi finanziari), se non è possibile minimizzare il rischio.
Questo è un piano che entra in gioco solo nel momento in cui il rischio si manifesta e serve per evitare che lo stesso crei troppi danni al progetto.

\subsection{Risk monitoring}
I rischi vanno costantemente monitorati per essere pronti ad intervenire qualora si manifestassero ed ancor prima per prevenirli nel caso in cui aumentasse la probabilità del loro manifestarsi.\\
Per monitorare i rischi si possono utilizzare degli indicatori di rischio, indicatori quantitativi che permettono di capire quando un rischio si sta avvicinando.
Ad esempio, se si è in ritardo in ogni consegna, vuol dire che si sono stimati male i tempi di sviluppo e di consegna.\\
Il monitoraggio va fatto ovviamente durante lo sviluppo del progetto e non alla fine così da avere il tempo di risolvere l'eventuale problematica venutasi a creare.

\subsection{Gestione del personale}
Gli ingegneri del software sono molto costosi poiché sono pochi e la richiesta è molto alta.\\
La gestione scorretta del personale porta inevitabilmente al fallimento del progetto.\\
Uno degli obiettivi è dunque quello di scegliere un gruppo di perone abbastanza coeso in modo da favorire l'apprendimento tra le persone del team possano e lo stimolo a vicenda a creare software di qualità.\\
Per comporre il team in modo corretto bisogna conoscere le personalità dei dipendenti ed abbinarle in modo bilanciato.\\
Alcune delle personalità sono:
\begin{itemize}[noitemsep]
    \item \textbf{Task-oriented}: trovano motivazioni nel lavorare e sono perfezioniste. Sono persone che lavorano bene anche da sole. La maggior parte degli ingegneri del software ha questa personalità
    \item \textbf{Self-oriented}: lavorano per raggiungere degli obiettivi personali che sono esterni al lavoro (es. la volontà di avere una bella macchina, casa, poter viaggiare...). Queste persone, tipicamente, vorrebbero avere il comando
    \item \textbf{Interaction-oriented}: sono stimolate ad andare al lavoro per le interazioni sociali. Il problema di queste persone potrebbe essere quello di perdere molto tempo a parlare e scambiare idee
\end{itemize}
Anche per quanto riguarda l'assegnazione del personale e le modalità di pianificazione esiste il processo \textbf{plan-driven} cioè a piani.
Tipicamente, la gestione del progetto per piani consiste nel definire degli step che definiscono cosa deve essere fatto, a chi deve essere assegnato il task, quali task ci sono e quali sono i termini temporali dei task.\\
Questa è la modalità più facile che permettere di stimare i tempi e quindi capire se si è in ritardo rispetto a quanto pianificato oltre a dare la possibilità al management di prendere delle decisioni informate in base al livello di sviluppo del progetto.\\
È importante che vengano fatte delle review regolari per controllare se si è in linea con quanto pianificato ed in caso quali sono le motivazioni che hanno comportato il non rispetto della pianificazione.\\
Le attività del progetto vanno divise in task che contengono alcune informazioni tra cui il tempo necessario per essere eseguite, il giorno in cui la si può iniziare a sviluppare, eventuali dipendenze tra le task (cioè se delle task vanno eseguite prima e altre dopo) e l'effort necessario per portarle a termine.\\
Si possono anche definire delle \textbf{milestone} all'interno del progetto cioè punti in cui un particolare obiettivo dovrebbe essere stato raggiunto.\\
Il momento del raggiungimento della milestone permette di avere una chiara idea sul rispetto delle tempistiche pianificate.
Lo stesso principio può essere applicato ai \textbf{deliverable} cioè tutti quei documenti o demo che possono essere consegnati al cliente.

La pianificazione del progetto può essere rappresentata in una tabella, dove ogni task è una riga ed ogni colonna rappresenta sia l'effort che la durata in giorni che eventuali dipendenze.\\
La rappresentazione migliore per la progettazione del progetto è il diagramma di \textbf{Gantt}, diagramma che rappresenta il flusso temporale del progetto.
Sulle ascisse è rappresentato il tempo, dal giorno zero al termine mentre sulle ordinate sono rappresentate le task.
Nel diagramma è presente una barra continua per ogni task, la barra inizia il giorno in cui la task dovrebbe cominciare e termina il giorno in cui dovrebbe finire perciò la lunghezza rappresenta la durata.\\
Sono rappresentate anche le milestone con dei rombi e spesso si trovano all'inizio o alla fine delle task e si possono rappresentare esplicitamente o meno anche le dipendenze con delle frecce che iniziano dalle task che sono la dipendenza e terminano nella task che richiede la dipendenza.\\
Questo diagramma permette di ragionare su come si possono ordinare le task per liberare risorse ed utilizzarle in modo migliore e più efficiente.\\
Un altro diagramma meno utilizzato è lo \textbf{staff-allocation} che, una volta fatto il digramma di Gantt, permette di allocare le persone alle varie task.
Anche in questo caso l'asse delle ascisse rappresenta il flusso temporale e quello delle ordinate rappresenta le persone.\\
La barra piena rappresenta una persona allocata full-time mentre la barra a metà rappresenta una persona allocata part-time.
Questo diagramma permette di capire quanto stanno lavorando i dipendenti e quando entrano ed escono dal progetto.\\
Questo diagramma fornisce anche una visuale diversa sulle relazioni di dipendenza perché rende evidente quante e quali persone stanno lavorando ad una task, per quanto tempo e cosa provocherebbe un ritardo in una task per le task successive.\\
Questo tipo di gestione è utilizzata anche su progetti molto lunghi (anche di anni) dove spesso la progettazione definita all'inizio non è rispettata del tutto e dovrà quindi essere cambiata ed adattata nel tempo.

L'alternativa alla gestione a piani di un progetto è la gestione agile che ricalca un po' l'idea dello sviluppo agile del processo.
Il sistema agile pone meno enfasi sulla progettazione dettagliata e più enfasi sulla mutabilità del progetto rispetto al cambiamento delle condizioni esterne.
È sempre presente una fase di \textbf{release planning} in cui si pianificano le release ed un'\textbf{allocation planning} che ha un tempo molto più corto di esecuzione.\\
Per questo tipo di pianificazione si usa il cosiddetto \textbf{planning game} che ha alcune fasi.
Si inizia con l'identificazione di storie che rappresentano le feature e ad ognuna di esse si assegna l'effort che serve per portare a compimento la storia così da poter stimare anche l'effort complessivo dell'intero progetto.
Dopodiché bisogna pianificare le release scegliendo in che ordine vanno implementate le feature e il tempo totale di ogni ciclo.\\
Si utilizza la velocity per capire quale delle storie inserire nella release ed all'inizio di ogni iterazione viene assegnata l'allocazione delle task in cui gli sviluppatori scelgono su quali task lavorare.\\
La scelta è fatta su base volontaria ma tutte le task devono essere implementate per completare le storie.\\
Circa a metà dell'iterazione è consigliabile fare una review per valutare se si è in orario con lo sviluppo o se si è in ritardo e nel caso riorganizzare le task rimanenti per poter consegnare un prodotto stabile seppur parziale al termine delle release.
In questo caso il termine \virgolette{task} è diverso da quello usato precedentemente nel diagramma Gantt perché in questo caso le task sono precise e non generiche.\\
La scadenza non può mai essere spostata quindi l'unica possibilità in caso di ritardi è la riduzione delle storie o task incluse nella release.\\
Questo planning ha dei vantaggi tra cui l'incentivare la comunicazione e la conoscenza delle dipendenze, la possibilità di scelta delle proprie task per i programmatori che quindi sentiranno un senso di appartenenza alla task e saranno più motivati a portarle a termine.\\
Tuttavia se il cliente non è disponibile a partecipare al planning game sarà difficile implementare questa metodologia.
Inoltre la metodologia agile funziona bene dove la comunicazione può essere informale e quindi è più faticosa da utilizzare in team numerosi o molteplici.\\
